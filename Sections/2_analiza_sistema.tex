
\section{Analiza sistema}

	U okviru poslovanja ne spada samo dostava namirnica do klijenta već i nabavka namirnica od snabdevača i njihovo skladištenje po magacinima. U nastavku opisujemo aktere koje prepoznajemo i koje su njihove uloge u sistemu.

\subsection{Akteri}
	\begin{itemize}
		\item{\textbf{Klijent}} - predstavlja krajnjeg korisnika. Klijent nakon registracije postavlja inicijalna podešavanja profila: tip obroka, broj porcija i broj obroka po nedelji. Na osnovu ovih podataka formira se cena pretplate i ova podešavanja se pamte. Klijent ima mogućnost menjanja ovih podataka u budućnosti i svaki put se cena formira na osnovu njih. U okviru podešavanja inicijalnih podataka unosi podatke o dostavi i načinu plaćanja, nakon čega bira obroke. Na osnovu izabranih obroka formira se spisak potrebnih namirnica sa njihovom količinom koje će mu biti dostavljene. Takođe, ovi podaci mogu biti naknadno izmenjeni. Klijent može u bilo kom trenutku otkazati pretplatu i preskočiti dostavu za narednu nedelju. U sklopu paketa dostavljaju se i recepti za obrok, napisani korak po korak. Klijentu se svake nedelje menja ponuda mogućih obroka. Sistem je zaslužan za pravljenje ponude i to radi po određenom algoritmu koji bira obroke i njihove recepte iz baze podataka. Klijent može odabrati datum i vreme dostave, međutim vreme dostave je podložno promeni u periodu do sat vremena od unete satnice, zbog dostupnosti dostavljača.
		\item{\textbf{Dostavljač}} - preuzima pakete sa liste isporuka za taj dan. U slučaju kašnjenja dostave obaveštava klijenta.
		\item{\textbf{Magacioner}} - pakuje pakete koji treba da se dostave u toku nedelje. Prima dostavljene namirnice i raspoređuje ih po magacinu.
		\item{\textbf{Koordinator}} - zadužen je za kontrolu magacina, koliko i kojih namirnica dolazi i odlazi iz magacina. U slučaju nedostatka određenih namirnica mora da kontaktira snabdevače kako bi ih nabavio. Takođe zadužen je za dodeljivanje dostavljača porudžbinama na osnovu dostupnosti dostavljača. Sistem pruža informaciju o nedostupnosti dostavljača.
		\item{\textbf{Snabdevač}} - proizvodi određene namirnice i u obavezi je da dostavi koordinatoru određenu količinu namirnica na njegov zahtev. Ima pristup sistemu, ali ne predstavlja zapošljeno lice u kompaniji ChooseFresh.
		\item{\textbf{Administrator}} - objavljuje konkurs za snabdevače, pravi ugovor o saradnji sa snabdevačima i zapošljava nova lica. 
		
		U daljem tekstu pod terminom \textit{zapošljena lica} podrazumevamo dostavljača, magacionera i koordinatora, dok pod terminom \textit{korisnici} podrazumevamo sve aktere u sistemu. Na slici \ref{fig:context_diagram} prikazan je dijagram konteksta na kome su predstavljeni svi akteri sistema.
		
				
	\end{itemize}
\subsection{Dijagram toka podataka}

\begin{figure}[H]
	\begin{center}
		\includegraphics[width=\textwidth]{context_diagram.png}

	    \caption{Dijagram konteksta}
	\label{fig:context_diagram}
    \end{center}
    
\end{figure}

	Na slici \ref{fig:DFD} prikazan je dijagram toka podataka nivoa 0. U nastavku su ukratko objašnjeni procesi predstavljeni dijagramom.
	
	
	\textbf{Upravljanje nalozima}: Ovim procesom je obuhvaćena registracija klijenta i zaposlenih, kao i prijavljivanje korisnika sistemu. Uključuje ažuriranje i brisanje naloga klijenta, zaposlenih i snabdevača.
	
	\textbf{Odabir i ažuriranje plana ishrane}: Podrazumeva inicijalni odabir plana ishrane na koji se klijent pretplaćuje, kao i proces kasnijeg ažuriranja plana ishrane.
		
	\textbf{Dostava paketa}: Uključuje proces zakazivanja isporuka, pakovanja paketa i dostave paketa klijentu. 
	
	
	\textbf{Nabavka namirnica}: Koordinator naručuje namirnice od snabdevača koji ih doprema do magacina, zatim magacioner raspoređuje primljene namirnice i vodi evidenciju o primljenim namirnicama.
	
	
	\textbf{Prijava i registracija snabdevača}: Ovaj proces je drugačiji od registracije klijenta i zaposlenih jer zahteva dodatne podatke (izveštaj inspekcije, prijavu snabdevača dobijenu apliciranjem snabdevača, i dokumentaciju snabdevača).


\begin{figure}[H]
	\begin{center}
		\includegraphics[width=\textwidth]{dfd.png}

    		\caption{Dijagram toka podataka nivoa 0}
    \label{fig:DFD}
    \end{center}
 
\end{figure}

	
