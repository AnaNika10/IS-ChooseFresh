
\section{Analiza sistema}

	U sklopu poslovanja ne spada samo dostava namirnica do klijenta već i nabavka namirnica od snabdevača i njihovo skladištenje po magacinima. U nastavku opisujemo aktere koje prepoznajemo i koje su njihove uloge u sistemu.

\subsection{Akteri}
	\begin{itemize}
		\item{\textbf{Klijent}} - predstavlja krajnjeg korisnika. Klijent može naručiti određen broj obroka po nedelji u skladu sa njegovom izabranom ishranom. Naručivanje obroka se vrši preko veb aplikacije i moguće je izmeniti odabir obroka do određenog dana u nedelji. U zavisnosti od broja obroka po nedelji i odabrane količine namirnica formira se plan na koji se klijent pretplaćuje. Klijent je takođe u mogućnosti da preskoči naručivanje obroka za neku od narednih nedelja. Uz svaki obrok dostavljaju se i recepti za obrok, napisani korak po korak. Klijentu se svake nedelje menja ponuda mogućih obroka. Sistem je zaslužan za pravljenje ponude i to radi po određenom algoritmu koji bira obroke i njihove recepte iz baze podataka.
		\item{\textbf{Dostavljač}} - preuzima paket sa identifikacionim brojem na osnovu kog dobija informacije kome i gde treba da dostavi paket sa namirnicama.
		\item{\textbf{Magacioner}} - pakuje pakete koji treba da se dostave u toku nedelje. Prima dostavljene namirnice i raspoređuje ih po magacinu.
		\item{\textbf{Koordinator}} - zadužen je za kontrolu magacina, koliko i kojih namirnica dolazi i odlazi iz magacina. U slučaju nedostatka određenih namirnica mora da kontaktira snabdevače kako bi ih nabavio.
		\item{\textbf{Snabdevač}} - proizvodi određene namirnice i u obavezi je da dostavi koordinatoru određenu količinu namirnica na njegov zahtev. Ima pristup sistemu, ali ne predstavlja zapošljeno lice u kompaniji HelloFresh.
		\item{\textbf{Administrator}} - objavljuje konkurs za snabdevače, skalpa ugovor o zaposlenju i pravi ugovor o saradnji sa snabdevačima.
		
		U daljem tekstu pod terminom \textit{zapošljena lica} podrazumevamo dostavljača, magacionera i koordinatora, dok pod terminom \textit{korisnici} podrazumevamo sve aktere u sistemu.
		
				
	\end{itemize}
\subsection{Dijagram toka podataka}

	% Dijagram konteksta
%	Od aktera sa sistemom interaguju: korisnik, dobavljac, snabdevac, magacioner, dostavljac, banka, HR. Od Skladista podataka imamo bazu podataka (D), ugovore o saradnji (M).
	% UBACITI DIJAGRAM
	% UBACITI DIJAGRAM
	\textbf{Registrovanje klijenta i odabir plana}: klijent se pri registraciji određuje plan na koji se pretplaćuje i obavezan je postupak kako bi mogao da pristupi uslugama aplikacije. Registraciju obavlja sam i kao odgovor dobija poruku da li se uspešno registrovao ili ne.
	
	\textbf{Prijavljivanje korisnika}: Da bi korisnik mogao da pristupi aplikaciji mora biti prijavljen. Ovaj postupak je identičan za sve aktere.
		
	\textbf{Odabir obroka}: Klijent bira obroke za narednu nedelju i potvrđuje svoj odabir.
	
	
	\textbf{Dostava paketa}: Dostavljač mora biti prijavljen da bi mogao da dobije informacije o paketu koji treba da dostavi. 
	
	
	\textbf{Poručivanje namirnica}: Koordinator pri prijavi na sistem dobija informacije o narudžbinama koje prosleđuje magacionerima. Od magacionera može da dobije obaveštenje da li fali neki proizvod i koordinator tada kontaktira snabdevače da organizuje prevoz robe do magacina. 
	
	\textbf{Pakovanje paketa}: Magacioner dobija spisak nespakovanih porudžbina i bira jednu. Pakuje potrebne namirnice i obeležava paket identifikacionim brojem, zatim preuzima sledeću narudžbinu.
	
	\textbf{Isporučivanje namirnica}: Snabdevači su obavezi da isporuče određen proizvod u određenoj količini na zahtev koordinatora. 
	% dodati proces registracije snabdevača (sklapanje ugovora)
	