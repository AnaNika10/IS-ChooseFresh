\subsubsection{Registrovanje snabdevača}

\begin{itemize}
	\item Kratak opis:
		\begin{itemize}
			\item Administrator pravi nalog novom snabdevaču.
		\end{itemize}
	\item Učesnici:
		\begin{itemize}
			\item Novi snabdevač
			\item Administrator
		\end{itemize}				
	\item Preduslovi:
		\begin{itemize}
		    \item Sistem je u funkciji
		\end{itemize}
	\item Postuslovi:
		\begin{itemize}
			\item U bazu je ubačen novi snabdevač i omogućeno mu je da pristupa novonapravljenom nalogu.
		\end{itemize}		
	\item Osnovni tok:
		\begin{enumerate}
		    \item Administrator dobija od inspekcije izveštaj i propratnu dokumentaciju.
		    \item Ukoliko je inspekcija uspešno prošla administrator pristupa sistemu i bira opciju za generisanje ugovora. 
		    \item Administrator unosi podatke o novom snabdevaču. 
		    \item Sistem generiše ugovor.
		    \item Administrator šalje mejl novom snabdevaču koji sadrži ugovor i zahtev da ga potpiše.
		    \item Novi snabdevač potpisuje i šalje ga nazad.
		    \item Administrator bira opciju za pravljenje novog naloga i pravi novi nalog novom snabdevaču
		    \item Administrator u bazu ubacuje njegove podatke iz prijave i arhivira njegovu dokumentaciju i izveštaj inspekcije.
		    \item Sistem šalje mejl novom snabdevaču o uspešnoj registraciji i pravilima korišnjenja naloga.
		\end{enumerate}
	\item Alternativni tok:
		\begin{itemize}
			 \item[2.a] Snabdevač ne ispunjava zahtevane uslove i njegova prijava je odbijena. Ne može podneti novu prijavu u narednih 6 meseci. Slučaj korišćenja se završava.
		  %  \item[4.a] Snabdevač nije pripremio potrebnu dokumentaciju prilikom inspekcije. U tom slučaju određuje se novi termin za inspekciju. Slučaj korišćenja se nastavlja od koraka 4.
	
		\end{itemize}
	\item Dodatne informacije:
		\begin{itemize}
			\item Neophodna dokumenta koje snabdevač pokazuje prilikom inspekcije su izvod iz Registra poljoprivrednih gazdinstava ili izvod iz Agencije za privredne registre.
		\end{itemize}						
\end{itemize}