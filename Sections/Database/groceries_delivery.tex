\subsection{Podaci o namirnicama i dostavi}

\textbf{\large Namirnice}
\vspace{0.3cm}

Klasa \textit{Namirnice} predstavlja jednu namirnicu i njenu dostupnu količinu u magacinu.

Atributi:
\begin{itemize}
    \item id - jedinstveni identifikator namirnice (PK)
    \item naziv 
    \item spisakSnabdevača - spisak svih snabdevača koji imaju ovu namirnicu u ponudi. Elementi liste su id snabdevača (SK)
    \item količina
\end{itemize}

\textbf{\large Nabavka}
\vspace{0.3cm}

Klasa \textit{Nabavka} predstavlja jednu nabavku namirnica koju zakazuje koordinator sa snabdevačem.

Atributi:
\begin{itemize}
    \item id - jedinstveni identifikator nabavke (PK)
    \item idMagacionera - magacioner koji prima isporuku (SK)
    \item idKoordinatora - koordinator koji je zakazao nabavku (SK)
    \item idSnabdevača - snabdevač koji dostavlja namirnice (SK)
    \item namirnice - spisak namirnica sa njihovim količinama koji se nalaze u jednoj nabavci. Prvi element para je id namirnice  (SK).
\end{itemize}

\textbf{\large Dostavljanje paketa}
\vspace{0.3cm}

Klasa \textit{Dostavljanje paketa} predstavlja jednu dostavu paketa  koji je klijent poručio.

Atributi:
\begin{itemize}
    \item id - jedinstveni identifikator paketa (PK)
    \item status - status paketa opisan u \ref{fig:StatePackage}
    \item idPorudžbine - porudžbina koja treba da se upakuje i dostavi (SK)
    \item idMagacionera - magacioner koji pakuje porudžbinu (SK)
    \item idDostavljača - dostavljač koji je zadužen za dostavu paketa do klijenta (SK)
\end{itemize}